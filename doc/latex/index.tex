\hypertarget{index_intro_sec}{}\section{Introduction}\label{index_intro_sec}
Dans le cadre de l\textquotesingle{}UE Programation Impérative du deuxième semestre, nous devions réaliser un projet dont le thème était Snake vs Schlanga. Nous avons donc créé une variante du jeu Snake disponible sur les anciens téléphones Nokia. Ainsi, il y a maintenant plusieurs serpents qui s\textquotesingle{}affrontent sur un plateau de jeu où la mort est sans pitié.

~\newline
 \hypertarget{index_install_sec}{}\section{Installation}\label{index_install_sec}
\hypertarget{index_step1}{}\subsection{Jeu}\label{index_step1}
Pour jouer au jeu, il suffit d\textquotesingle{}exécuter le fichier play\+\_\+game.\+sh et une fenêtre s\textquotesingle{}ouvre après quelques instants \+: c\textquotesingle{}est parti !\hypertarget{index_step2}{}\subsection{Tests unitaires}\label{index_step2}
Pour lancer les tests unitaires, il suffit d\textquotesingle{}exécuter le fichier run\+\_\+tests.\+sh. Vous devez entrer votre mot de passe car le fichier tente d\textquotesingle{}installer la librairie cmocka que nous avons utilisé pour créer ces tests unitaires. Après quelques instants, vous pourrez voir le résultat des tests unitaires.

~\newline
 \hypertarget{index_jeu}{}\section{Principe du jeu}\label{index_jeu}
\hypertarget{index_jeu_base}{}\subsection{Principes de base}\label{index_jeu_base}
Le joueur doit tenter de survivre le plus longtemps possible sur le plateau de jeu. Pour se déplacer, le joueur peut utiliser les flèches directionnelles ou les touches q et d. Attention, la direction est en fonction de la tête du serpent, ce qui signifie que si le joueur appuie sur la flèche droite, le serpent tourne à droite. Les murs sont mortels, la taille du plateau est fixe tout au long de la partie.\hypertarget{index_jeu_ia}{}\subsection{Intelligences artificielles}\label{index_jeu_ia}
Les serpents non contrôlés par le joueur sont régis par deux types d\textquotesingle{}intelligence artificielle \+:
\begin{DoxyItemize}
\item Une intelligence artificielle aléatoire \+: celle-\/ci teste uniquement la prochaine case et évite les collisions. Ses déplacements sont gérés de manière aléatoire.
\item Une intelligence artificielle défensive \+: celle-\/ci compte le nombre d\textquotesingle{}obstacles dans un carré de 5x5 à gauche, devant et à droite du snake. Ensuite, le serpent va dans la direction dans laquelle il y a le moins d\textquotesingle{}obstacles.
\end{DoxyItemize}

~\newline
 \hypertarget{index_options}{}\section{Options}\label{index_options}
Lors du lancement du jeu, le joueur a accès à un menu d\textquotesingle{}options accessible en appuyant sur la touche \char`\"{}o\char`\"{}. Dans ce menu, le joueur peut modifier de nombreux paramètres du jeu \+:
\begin{DoxyItemize}
\item La vitesse des serpents \+: lent, moyen ou rapide
\item Le nombre de serpents sur le plateau \+: de 1 à 4
\item Le type de plateau \+: normal, 1 contre 1 (2 serpents uniquement) ou grand
\item L\textquotesingle{}existence de murs au milieu du plateau 
\end{DoxyItemize}